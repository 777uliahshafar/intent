%%%%%%%%%%%%%%%%%%%%%%%%%%%%%%%%%%%%%%%%%
% Simple Article
% Integrated article template with simple for make4ht
% LaTeX Class
% Version 1.0 (10/11/20)
%
% This class originates by:
% Vel and  Nicolas Diaz
%
% Authors:
% Muhammad Uliah Shafar
%
%
% Free License:
%
%
%%%%%%%%%%%%%%%%%%%%%%%%%%%%%%%%%%%%%%%%%
\documentclass[12pt]{simart} % Font size (can be 10pt, 11pt or 12pt)

%----------------------------------------------------------------------------------------
%	TITLE SECTION
%----------------------------------------------------------------------------------------
% MAIN TITLE SECTION
\title{
\textbf{Letter Of Intent}\\
{Turkiye Scholarship} \\
} % Title and subtitle
%\date{\textbf{\DTMtoday}}
\date{\textbf{\today}}
\author{Uliah Shafar}

%----------------------------------------------------------------------------------------
% OTHER TITLE SECTION

%\title{\textbf{Sistem Sarana dan Prasarana Jl. Pinggir Laut} \\ {\Large\itshape Infrastructure of Waterfront Parepare City}} % Title and subtitle

%\author{\textbf{Uliah Shafar} \\ \textit{Universitas Diponegoro}} % Author and institution

%\date{\today} % Date, use \date{} for no date

%----------------------------------------------------------------------------------------



\begin{document}

\maketitle % Print the title section

%----------------------------------------------------------------------------------------
%	ABSTRACT AND KEYWORDS
%----------------------------------------------------------------------------------------
\vspace{30pt} % Vertical whitespace between the abstract and first section

%----------------------------------------------------------------------------------------
%	ESSAY BODY
%----------------------------------------------------------------------------------------
\section{Letter of Intent}
%Architecture has been a major influence in my life. It is quite around a decade now. It started when I was in vocational high school and continued in master's degree.
%What has become a great impact was when I had learnt how architecture from around the world shape the life of today society.
%When I was in undergraduate, we often explore world architecture, like from turkiye, europe and even asian countries such as korea.
%This had became a reflection to my country. For example,

After I graduated a master from one of the top ten Indonesia's University with cumlaude predicate, I came back to my hometown which is quite far away from my university.
While I was on my way home, I passed by my vocational high school.
In this school, I had experienced working together on drawing assignment with my peers.
Beside that, I had also did an internship in the biggest cement company in the south sulawesi. All of these knowledge and experiences were the beginning from my journey to architecture field which I have always been passionate about.

Not long after a high school farewell trip, I continued to study architecture in undergraduate's program at University Technology of Yogyakarta.








%These experiences was not only the reason (or tipping point) I have been continuing on this arena until now, but they also build my character, work ethic, team work and responsibility.


\begin{comment}

Preferensi Pengunjung terhadap Ruang Publik di Kawasan Tepi Laut Senggol Parepare
the use effectiveness of the square as a public open space

village's mosque

The village was located in suburb of Yogyakarta region, called a parangritis village. It was southern of Yogyarkarta which was around 20 km away.
Every weekend in a month, I participated constructing a small library for the Mosque. We began it with creating its shelves and we ended it with cleaning up the mosque's surrounding.
We had had quite numbers of discussions about the plan to go there. There were around 15 of us who were Architecture students from any university in Yogyarkarta city. The voluntary project was successfully completed on time.

A workshop about scientific writing held by Science and technology Faculty of University technology of Yogyakarta,

DED workshop by my university which indicated that I was passed for Detailed Engineering Design (DED) course. It had used for continuing the next semester which full of project that required DED Skill.

Although this activity was outside my scope, I have volunteered following my passion to study abroad. I have gained several knowledge as I helped this company, especially on know-how register in oversea university.

Organizer on Education Expo

This seminar was purposely telling the importance of preparation of the ASEAN Economic Community (AEC). The seminar has many magnificent popular speaker from Indonesia. One of them was working internationally.

Visitor Preference on Public Space in Senggol Waterfront Parepare

Preference of visitor towards public space in Senggol Waterfront Parepare

Preference of visitor towards public space in Senggol Waterfront P


\newpage
\begin{spacing}{1.5}

\noindent {\Large Penerima:\\
Ibu Wijayanti\\
087832827227\\
\textbf{Jl.  Puri I B2/19, Puri Asri Perdana, Banyumanik,\\
Semarang, 50266}\\}

\end{spacing}

\begin{spacing}{1.5}
\noindent {\Large Pengirim:\\
M. Uliah Shafar\\
081355190767\\
\textbf{Jl.  Handayani No.7, rt. 2, rw. 6, Lapadde, Ujung\\
Parepare, 91112}\\}

\end{spacing}

\end{comment}











%\begin{figure}[htbp]
%\centering
%\begin{subfigure}{6cm}
%\centering\includegraphics[width=5cm]{placeholder.jpg}
%\caption{}
%\end{subfigure}%
%\begin{subfigure}{6cm}
%\centering\includegraphics[width=5cm]{placeholder.jpg}
%\caption{}
%\end{subfigure}\vspace{10pt}
%
%\caption{Lorem Ipsum}
%\label{}
%\end{figure}

%----------------------------------------------------------------------------------------
%	BIBLIOGRAPHY
%----------------------------------------------------------------------------------------





\end{document}
